%   This is the template for the Smart Contract Lab.

%   If you have any questions you can reach out to me:
%   lazaronicolas.hofmann@uzh.ch

%   This file serves as an orientation giude for anyone deciding to use
%   this template. Read this before making any changes to the other files.

%%%%%%%%%%%%%%%%%%%%%%%%%%%%%%%%%%%%%%%%%%%%%%%%%%%%%%%%%%%%%%%%%%%%%%%%%%

                            %   STRUCTURE   %

%   The structure and sequence is defined in main.tex

%%%%    CHAPTER
    
%%%%    To add new chapters you need to first \input{chapter_file_name}
%%%%    in main.tex and start a new chapter inside chapter_file_name.tex
%%%%    using \chapter{Chapter Title}

%%%%%%%%%   SECTION

%%%%%%%%%   To add a new section use \section{Section Title} under the
%%%%%%%%%   corresponding chapter

%%%%%%%%%%%%%   SUB-SECTION

%%%%%%%%%%%%%   To add a sub-section use \subsection{Sub-Section Title}
%%%%%%%%%%%%%   under the corresponding section


%%%%%%%%%%%%%%%%%%%%%%%%%%%%%%%%%%%%%%%%%%%%%%%%%%%%%%%%%%%%%%%%%%%%%%%%%%

                            %   TITLEPAGE   %

% The title is modifyable inside the file 0a_titlepage.tex


%%%%%%%%%%%%%%%%%%%%%%%%%%%%%%%%%%%%%%%%%%%%%%%%%%%%%%%%%%%%%%%%%%%%%%%%%%

                            %   ABSTRACT   %

% The abstract is modifyable inside the file 0b_abstract.tex


%%%%%%%%%%%%%%%%%%%%%%%%%%%%%%%%%%%%%%%%%%%%%%%%%%%%%%%%%%%%%%%%%%%%%%%%%%

                        %   TABLE OF CONTENTS   %

% The TOC is automatically updated


%%%%%%%%%%%%%%%%%%%%%%%%%%%%%%%%%%%%%%%%%%%%%%%%%%%%%%%%%%%%%%%%%%%%%%%%%%

                            %   FIGURES   %

% The list of figures is automatically updated

% To add a figure use the following command:

\begin{figure}[H]
    \centering
    \captionsetup{font=bf}
    \caption{Blockchain data structure} % Title of Figure %
    \includegraphics[width=14cm]{figures/blockchain.png} % Path to image %
    \label{fig:myfigure} % % must be unique % %
    \begin{tablenotes}
        \item \textit{Source:} Raikwar et al. (2019, p. 148553) % Source %
    \end{tablenotes}
\end{figure}

% The file for the figure must be stored in the folder figures and
% therefore the image must be defined as follows: figures/figure_file


%%%%%%%%%%%%%%%%%%%%%%%%%%%%%%%%%%%%%%%%%%%%%%%%%%%%%%%%%%%%%%%%%%%%%%%%%%

                            %   TABLES   %

% The list of tables is automatically updated

% To add a table use the following command:

\begin{table}[H]
    \centering
    \captionsetup{font=bf}
    \caption{Some random numbers} % Title of Table %
    \begin{tabular}{|c|c|c|c|c|c|} % Define Number of Columns %
        \hline
        \textbf{AA} & \textbf{BB} & \textbf{CC} & \textbf{DD} & \textbf{EE} & \textbf{FF} \\ % First Row %
        \hline
        1 & 2 & 3 & 4 & 5 & 6 \\ % Second Row %
        \hline
    \end{tabular}
    \label{tab:mytable}
    \begin{tablenotes}
        \centering
        \item \textit{Source:} Backes-Gellner et al. (2001) % Source %
    \end{tablenotes}
\end{table}


%%%%%%%%%%%%%%%%%%%%%%%%%%%%%%%%%%%%%%%%%%%%%%%%%%%%%%%%%%%%%%%%%%%%%%%%%%


                            %   ABBREVIATIONS   %

%   Edit and add abbreviations in the 0c_abbreviations.tex file


%%%%%%%%%%%%%%%%%%%%%%%%%%%%%%%%%%%%%%%%%%%%%%%%%%%%%%%%%%%%%%%%%%%%%%%%%%
                            %   CITATIONS   %

% The bibliography is automatically and alphabetically updated

% To reference a source you must find the BibTeX of a scientific paper

%%% Example: Google Scholar

%%% 1.  Look for Paper
%%% 2.  Press "Cite" / "Zitieren"
%%% 3.  Press "BibTeX"
%%% 4.  Copy BibTeX-string
%%% 5.  Paste string to sources.bib
%%% 6.  Copy identifier after @article / @inproceedings etc.                      e.g.,  "adam2020blockchain"
%%% 7.  Go to chapter where you want to cite
%%% 8.  a) Parenthical: \citep{adam2020blockchain}
%%%     b) Narrative:   \citet{adam2020blockchain}


%%%%%%%%%%%%%%%%%%%%%%%%%%%%%%%%%%%%%%%%%%%%%%%%%%%%%%%%%%%%%%%%%%%%%%%%%%

                            %   AUTHORS   %

% Authors must be manually updated using the following command. Leave the
% format in 10_Xauthors.tex untouched

\begin{tikzpicture}[remember picture,overlay]
    \node[circle,clip,inner sep=-36pt] at ([xshift=4.8cm, yshift=-4.8cm]current page.north west)
    {\includegraphics[width=3cm, trim=0 0 0 -560pt, clip]{portraits/portrait1.jpg}}; % % % % % Define File here % % % % %
    \node[below=1.75cm, align=center, font=\bfseries] at ([xshift=4.8cm, yshift=-4.8cm]current page.north west)
    {John Doe}; % % % % % Change Name here % % % % %
    \node[below=2.5cm, align=center, font=\bfseries] at ([xshift=4.8cm, yshift=-4.8cm]current page.north west)
    {Manager}; % % % % % Change Function here % % % % %
\end{tikzpicture}

% To insert the portrait image upload the image to portraits folder. To
% define the path use portraits/file_name

% % % Please note, that you must use the same jpg-file sent to you by
% % % Daniel in order to have the right cropping and clipping. The size
% % % is around 2MB. Or find it on Teams
% % % Publication/2_Video&Visuals/Profile pictures.

% There are two author files. One is for a team of five and the other for
% a team of six members. whichever you need, you must define it in the
% main.tex file


%%%%%%%%%%%%%%%%%%%%%%%%%%%%%%%%%%%%%%%%%%%%%%%%%%%%%%%%%%%%%%%%%%%%%%%%%%